\documentclass{article}
\usepackage[utf8]{inputenc}
\usepackage{amsmath}
\DeclareMathOperator{\sgn}{sgn}
\usepackage{hyperref}
\usepackage{enumitem}
\usepackage{commath}
\usepackage[margin=0.5in]{geometry}
\title{Spline Math}
\author{James Kuszmaul}
\begin{document}

\maketitle

\section{Introductions}

This document describes some of the math going on in computing constraints on
the maximum velocities in \texttt{trajectory.*}. Some of the thoughts are
somewhat disorganized and represent a stream-of-consciousness more than actually
useful documentation.

For a given path, we will have:

\begin{tabular}{r|l}
Name & Description \\ \hline
$v_l$, $v_r$ & Left and right wheel velocities \\
$\theta$ & Robot heading/yaw at a position along the path \\
$r$ & Robot radius (assuming center of mass in center of robot) \\
$u_l$, $u_r$ & Left and right wheel voltages \\
$V_{max}$ & Maximum allowable voltage (typically 12V) \\
$v_{max}$ & User-specified maximum velocity \\
$u$ & $\begin{bmatrix} u_l \\ u_r \end{bmatrix}$ \\
$A$, $B$ & LTI, matrices such that $\begin{bmatrix} \dot{v}_l \\ \dot{v}_r
\end{bmatrix} = A \begin{bmatrix} v_l \\ v_r \end{bmatrix} + B u$ \\
$s$ & Distance along the path \\
\end{tabular}

We are interested in discovering the fastest way to traverse the path in
question, given constraints on:
\begin{itemize}
\item Friction limits of the wheels (assumed to static and time/velocity
      invariant).
\item Voltage limits on the motors.
\item Maximum velocity.
\end{itemize}

The voltage constraints are essentially provided by the linear dynamics shown
above:

\begin{align}
\begin{bmatrix} \dot{v}_l \\ \dot{v}_r \end{bmatrix} &=
   A \begin{bmatrix} v_l \\ v_r \end{bmatrix} + B u
\end{align}

Also, at any given point we must maintain velocities/accelerations that actually
allow us to follow the path (as opposed to optimizing $v_l$/$v_r$ separately...).

Using the velocities of the individual wheels as a starting point, and
introducing dummy variables for the velocity/acceleration of the center of the
robot (corresponding to time derivatives of $s$):

\begin{align*}
v_c &= \frac{v_l  r_r + v_r  r_l}{r_r + r_l} \\
\begin{bmatrix} v_l \\ v_r \end{bmatrix} &=
    \begin{bmatrix} 1 - r_l \frac{d\theta}{ds} \\ 1 + r_r \frac{d\theta}{ds}
\end{bmatrix} v_c \\
\begin{bmatrix} \dot{v}_l \\ \dot{v}_r \end{bmatrix} &=
  \begin{bmatrix} 1 - r_l \frac{d\theta}{ds} \\ 1 + r_r \frac{d\theta}{ds}
  \end{bmatrix} \dot{v}_c +   \begin{bmatrix} - r_l \frac{d^2\theta}{ds^2} v_c
  \\  r_r \frac{d^2\theta}{ds^2} v_c \end{bmatrix} v_c \\
\begin{bmatrix} \dot{v}_l \\ \dot{v}_r \end{bmatrix} &=
  \begin{bmatrix} 1 - r_l \frac{d\theta}{ds} \\ 1 + r_r \frac{d\theta}{ds}
  \end{bmatrix} \dot{v}_c +   \begin{bmatrix} -r_l \\ r_r \end{bmatrix} \frac{d^2\theta}{ds^2} v_c^2 \\
\end{align*}

And finally, we need to know the lateral accelerations of the two wheels as
part of determining whether we would be breaking friction. Note that, because of
the general assumptions we are making about the setup of the robot, the lateral
acceleration on the two wheels will be identical (if they were not identical,
that would imply the robot was tearing itself apart side-to-side).

\begin{align*}
a_{lat} &= \frac{d\theta}{ds} v_c^2 \\
\end{align*}

Finally, we need to be combine the lateral/longitudinal accelerations of each
wheel to determine whether the wheel would break friction. While technically
there is only one expression for this, as a practical matter we have a bit of
leeway in defining exactly what the relationship should be (e.g., if we want
separate max longitudinal/lateral accelerations, we can create an ellipse; we
could also just create a raw rectangle and decouple the two, although that would
probably be a particularly poor approximation of the actual dynamics). For now
we will say that our friction is limited by some nonnegative convex function
$g(a_{lat}, a_{lng}) \le 1$ for any given wheel.

Summarizing our equations and constraints, we have:

\begin{align*}
v_c &= \frac{v_l r_r + v_r r_l}{r_r + r_l} \\
a_{lat} &= \frac{d\theta}{ds} v_c^2 \\
g(a_{lat}, \dot{v}_l) &\le 1 \\
g(a_{lat}, \dot{v}_r) &\le 1 \\
\abs{u_l} &\le V_{max} \\
\abs{u_r} &\le V_{max} \\
\begin{bmatrix} v_l \\ v_r \end{bmatrix} &= \begin{bmatrix} 1 - r_l
    \frac{d\theta}{ds} \\ 1 + r_r \frac{d\theta}{ds}
\end{bmatrix} v_c \\
\begin{bmatrix} \dot{v}_l \\ \dot{v}_r \end{bmatrix} &=
   A \begin{bmatrix} v_l \\ v_r \end{bmatrix} + B \begin{bmatrix} u_l \\ u_r
   \end{bmatrix} \\
\begin{bmatrix} \dot{v}_l \\ \dot{v}_r \end{bmatrix} &=
  \begin{bmatrix} 1 - r_l \frac{d\theta}{ds} \\ 1 + r_r \frac{d\theta}{ds}
  \end{bmatrix} \dot{v}_c +   \begin{bmatrix} -r_l \\ r_r \end{bmatrix} \frac{d^2\theta}{ds^2} v_c^2 \\
\end{align*}

With $v_l$, $v_r$, $u_l$, and $u_r$ free at all $s$ except the start and end, at
which the velocities must
be zero (technically, we could make do with allowing non-zero velocities, but
then we could end up in situations that would be impossible to solve--because
the splines have differentiable curvature, any of the splines we use will always
have some valid velocity profile, but depending on the curvature, that velocity
may be arbitrarily small).

We can provide an upper limit on the goal velocity at any given point with the
min of the specified max velocity and the lateral-acceleration limited velocity.
This also gives an initial pass of the longitudinal accelerations that would be
required to follow this profile.

\begin{align*}
v_{lat} &= \min (v_{max}, \sqrt{\frac{a_{lat,max}}{\frac{d\theta}{ds}}}) \\
\dot{v}_{lat} &= \begin{cases} 0 & v_{lat} = v_{max} \\
-\frac12 \sqrt{\frac{a_{lat,max}}{\frac{d\theta}{ds}^3}} \frac{d^2\theta}{ds^2}
v_{lat} & v_{lat} \ne v_{max} \\
\end{cases} \\
&= \begin{cases} 0 & v_{lat} = v_{max} \\
-\frac12 \frac{a_{lat,max}}{\frac{d\theta}{ds}^2} \frac{d^2\theta}{ds^2}
 & v_{lat} \ne v_{max} \\
\end{cases} \\
\end{align*}

If we start to consider accounting also for longitudinal accelerations, we can
start with:

\begin{align}
\begin{bmatrix} \dot{v}_l \\ \dot{v}_r \end{bmatrix} &=
  \begin{bmatrix} 1 - r_l \frac{d\theta}{ds} \\ 1 + r_r \frac{d\theta}{ds}
  \end{bmatrix} \dot{v}_c +   \begin{bmatrix} -r_l \\ r_r \end{bmatrix} \frac{d^2\theta}{ds^2} v_c^2 \\
(r_r + r_l)
\left[ \begin{bmatrix} \dot{v}_l \\ \dot{v}_r \end{bmatrix} &=
  \begin{bmatrix} 1 - r_l \frac{d\theta}{ds} \\ 1 + r_r \frac{d\theta}{ds}
  \end{bmatrix} \frac{\dot{v}_l r_r + \dot{v}_r r_l}{r_r + r_l} +   \begin{bmatrix} -r_l \\ r_r \end{bmatrix} \frac{d^2\theta}{ds^2} v_c^2
  \right] \\
\begin{bmatrix}
r_l + r_l r_r \frac{d\theta}{ds} & -r_l + r_lr_l \frac{d\theta}{ds} \\
-r_r - r_r r_r \frac{d\theta}{ds} & r_r - r_r r_l \frac{d\theta}{ds} \\
\end{bmatrix}
\begin{bmatrix} \dot{v}_l \\ \dot{v}_r \end{bmatrix} &=
  \begin{bmatrix} -r_l \\ r_r \end{bmatrix} (r_l + r_r) \frac{d^2\theta}{ds^2} v_c^2 \\
\begin{bmatrix}
1 + r_r \frac{d\theta}{ds} & -1 + r_l \frac{d\theta}{ds} \\
-1 - r_r \frac{d\theta}{ds} & 1 - r_l \frac{d\theta}{ds} \\
\end{bmatrix}
\begin{bmatrix} \dot{v}_l \\ \dot{v}_r \end{bmatrix} &=
  \begin{bmatrix} -1 \\ 1 \end{bmatrix} (r_l + r_r) \frac{d^2\theta}{ds^2} v_c^2 \\
\begin{bmatrix}
1 + r_r \frac{d\theta}{ds} & -1 + r_l \frac{d\theta}{ds} \\
1 + r_r \frac{d\theta}{ds} & -1 + r_l \frac{d\theta}{ds} \\
\end{bmatrix}
\begin{bmatrix} \dot{v}_l \\ \dot{v}_r \end{bmatrix} &=
   -(r_l + r_r) \frac{d^2\theta}{ds^2} v_c^2 \\
(1 + r_r \frac{d\theta}{ds}) \dot{v}_l + (-1 + r_l \frac{d\theta}{ds}) \dot{v}_r &=
   -(r_l + r_r) \frac{d^2\theta}{ds^2} v_c^2 \\
%(1 + r \frac{d\theta}{ds}) \dot{v}_l + (-1 + r \frac{d\theta}{ds}) (2\dot{v}_c -
%\dot{v}_l) &=
%   -2 r \frac{d^2\theta}{ds^2} v_c^2 \\
%2 \dot{v}_l + 2(-1 + r \frac{d\theta}{ds}) \dot{v}_c
% &= -2 r \frac{d^2\theta}{ds^2} v_c^2 \\
%\dot{v}_l &= (1 - r \frac{d\theta}{ds}) \dot{v}_c - 2 r \frac{d^2\theta}{ds^2} v_c^2 \\
%\dot{v}_r &= (1 + r \frac{d\theta}{ds}) \dot{v}_c + 2 r \frac{d^2\theta}{ds^2} v_c^2 \\
\begin{bmatrix}
1 & 0 \\
0 & 1 \\
-1 & 0 \\
0 & -1 \\
\end{bmatrix}
\begin{bmatrix} \dot{v}_l \\ \dot{v}_r \end{bmatrix} &\le g'(v_c, s) \\
\begin{bmatrix}
1 & 0 \\
0 & 1 \\
-1 & 0 \\
0 & -1 \\
\end{bmatrix} \left(
\begin{bmatrix} 1 - r_l \frac{d\theta}{ds} \\ 1 + r_r \frac{d\theta}{ds}
\end{bmatrix} \dot{v}_c +   \begin{bmatrix} -r_l \\ r_r \end{bmatrix} \frac{d^2\theta}{ds^2} v_c^2
\right) &\le g'(v_c, s)
\end{align}

Note how we have a linear relationship between $\dot{v}_l$ and $\dot{v}_r$,
where the slope of the line between the two is constant at any given point on
the path, but the y-intercept and the max/min bounds get tighter with speed.
Furthermore, the velocity along the path ($v_c$) is a linear function of the
wheel velocities $v_l$ and $v_r$. Thus, we can state, at any given point on the
path, a lower velocity along the path will not decrease our acceleration
bounds (if we are travelling in a straight line, then our speed does not affect
the straight-line acceleration limits).

We also can introduce the voltage constraints:

\begin{align}
\begin{bmatrix} \dot{v}_l \\ \dot{v}_r \end{bmatrix} &=
   A \begin{bmatrix} v_l \\ v_r \end{bmatrix} + B \begin{bmatrix} u_l \\ u_r
   \end{bmatrix} \\
\begin{bmatrix} \dot{v}_l \\ \dot{v}_r \end{bmatrix} &=
   A \begin{bmatrix} 1 - r_l \frac{d\theta}{ds} \\ 1 + r_r \frac{d\theta}{ds}
     \end{bmatrix} v_c + B \begin{bmatrix} u_l \\ u_r \end{bmatrix} \\
\begin{bmatrix} 1 - r_l \frac{d\theta}{ds} \\ 1 + r_r \frac{d\theta}{ds}
  \end{bmatrix} \dot{v}_c +   \begin{bmatrix} -r_l \\ r_r \end{bmatrix} \frac{d^2\theta}{ds^2} v_c^2
  &= A \begin{bmatrix} 1 - r_l \frac{d\theta}{ds} \\ 1 + r_r \frac{d\theta}{ds}
     \end{bmatrix} v_c + B \begin{bmatrix} u_l \\ u_r \end{bmatrix} \\
\end{align}

With these constraints, we can no longer guarantee that the set of feasible
velocities at any given point along the path is contiguous. However, we can
guarantee that at any given point there will be at most 2 contiguous sets of
valid velocities---in the case where there are two separate sets of allowable
velocities, then one corresponds to the lower velocities which start at zero and
go up until we can no longer use our finite available voltage to track the path.
The second zone can occur at higher velocities where we are going so fast that
the deceleration from the Back-EMF of the motor is enough to slow us down, but
not so fast that we would lose friction.

This divergence does technically create a bit of a challenge--if it were to,
e.g., split into arbitrarily many possible paths than that would complicate the
optimization problem substantially. However, because:
\begin{enumerate}
\item We only split into two branches at once; and
\item we should only have a limited number of branchings for a given N-spline
\footnote{I'll leave this as an exercise to the reader/my future self. However,
logically, a spline with a given number of points can only have so many features
and so should be able to have a very finite number of branchings.}.
\end{enumerate}

It should not actually be an issue to exhaustively attempt to plan for all
possible branchings, knowing that the path of ``always follow the slower
branch'' will always provide a feasible path, and choose whatever feasible plan
is fastest.

Once we choose which branch to take, it is a relatively simple matter of
exercising the existing method. Essentially, taking the set of feasible
velocities/accelerations as a starting point we do a forwards and backwards pass
(order is unimportant)) where we start at our initial/terminal velocity, and
accelerate forwards/decelerate backwards as hard as possible for each pass. If
attempting to accelerate as hard as possible on the forwards pass (or,
conversely, decelerate on the backwards) results in us violating the feasible
velocity bounds, we can simply set the velocity at that point for that pass to
the highest feasible velocity. We can guarantee that the pass in the opposite
direction will \emph{not} get stuck at that same point because if that would
suggest we had different physics going backwards than forwards, which is not the
case\footnote{I'm too lazy to spell this logic out more carefully, and I don't
think it is wrong. That is not guaranteed, however.}.

\section{Specialization for Friction Circle}

As our particular friction circle, we will attempt to use a straight
ellipse\footnote{Note that this has numerical stability issues due to the
infinite slope near zero}.
With a $a_{lat,max}$ and $a_{lng,max}$ for the maximum lateral and longitudinal
accelerations, i.e.:

\begin{align*}
g(a_{lat}, a_{lng}) &= \left(\frac{a_{lat}}{a_{lat,max}}\right)^2
                       + \left(\frac{a_{lng}}{a_{lng,max}}\right)^2 \\
\end{align*}

We care about turning this into a function of the path geometry and the current
velocity that returns the max longitudinal acceleration and lets us calculate
the friction-limited maximum acceleration, $v_{c,fmax}$. In order to do so, we
note that, in order to be at the edge of a feasible velocity, we observe that,
at any given $v_c$, we will have two pairs of constraints on the center-of-mass
longitudinal acceleration that both must be met,
corresponding to the fact that the each wheel creates an upper/lower bound
on the valid accelerations of the center-of-mass of the robot, but at high
$\frac{d^2\theta}{ds^2}v_c^2$, the constraints from each wheel may not overlap.
Thus, we care about the $v_c$ such that the pairs of constraints \emph{just
barely} overlap. This will happen when one constraint from the right wheel and
one from the left are just on the edge of being violated; furthermore, of the
four possible left/right pairings (+/- of left/right), only two pairs correspond
to being on the edge of overlapping (the other two pairs correspond to, e.g.,
having the upper acceleration limit in the same spot, as occurs when driving
straight; in such a situation, there are plenty of valid accelerations, we just
happen to start slipping both wheels at the same time). Which pair to look at is
determined by the signs on $\dot{v}_c$--if they are the same (as would occur
when going nearly straight), then we care about when the opposite limits
overlap (e.g., upper limit on the left, lower on the right).

\begin{align*}
\dot{v}_l, \dot{v}_r &\le a_{lng, max} \sqrt{1 - \left(\frac{a_{lat}}{a_{lat,max}}\right)^2} \\
\dot{v}_l, \dot{v}_r &\le a_{lng, max} \sqrt{1 - \left(\frac{\frac{d\theta}{ds}v_c^2}{a_{lat,max}}\right)^2} \\
  \begin{bmatrix} 1 - r_l \frac{d\theta}{ds} \\ 1 + r_r \frac{d\theta}{ds} \\
 -1 + r_l \frac{d\theta}{ds} \\ -1 - r_r \frac{d\theta}{ds}
  \end{bmatrix} \dot{v}_c  &\le  \begin{bmatrix} r_l \\ -r_r \\ -r_l \\ r_r \end{bmatrix} \frac{d^2\theta}{ds^2} v_c^2
 + a_{lng, max} \sqrt{1 - \left(\frac{\frac{d\theta}{ds}v_c^2}{a_{lat,max}}\right)^2} \\
0 &= \pm \{r_l,r_r\} \frac{d^2\theta}{ds^2} v_{c,fmax}^2 +  a_{lng, max} \sqrt{1 -
  \left(\frac{\frac{d\theta}{ds}v_{c,fmax}^2}{a_{lat,max}}\right)^2} \\
\left(\frac{d^2\theta}{ds^2} \frac{\{r_l,r_r\}v_{c,fmax}^2}{a_{lng, max}}\right)^2 &=
  1 - \left(\frac{\frac{d\theta}{ds}v_{c,fmax}^2}{a_{lat,max}}\right)^2 \\
v_{c,fmax} &= \min_{r \in \{r_l, r_r\}}
\left(\left(\frac{d^2\theta}{ds^2} \frac{r}{a_{lng, max}}\right)^2 +
   \left(\frac{\frac{d\theta}{ds}}{a_{lat,max}}\right)^2\right)^{-\frac14} \\
\begin{bmatrix} 1 - r_l \frac{d\theta}{ds} \\ 1 + r_r \frac{d\theta}{ds}
\end{bmatrix} \dot{v}_c +   \begin{bmatrix} -r_l \\ r_r \end{bmatrix} \frac{d^2\theta}{ds^2} v_c^2
&= \begin{bmatrix}\pm 1 \\ \pm 1\end{bmatrix}
   a_{lng, max} \sqrt{1 - \left(\frac{\frac{d\theta}{ds}v_c^2}{a_{lat,max}}\right)^2} \\
K_2 \dot{v}_c + K_1 v_c^2
&= \begin{bmatrix}\pm 1 \\ \pm 1\end{bmatrix}
   a_{lng, max} \sqrt{1 - \left(\frac{\frac{d\theta}{ds}v_c^2}{a_{lat,max}}\right)^2} \\
K_2' K_2 \dot{v}_c + K_2' K_1 v_c^2
&= K_2' \begin{bmatrix}\pm 1 \\ \pm 1\end{bmatrix}
   a_{lng, max} \sqrt{1 - \left(\frac{\frac{d\theta}{ds}v_c^2}{a_{lat,max}}\right)^2} \\
K_2' K_1 v_c^2
&= K_2' \begin{bmatrix}\pm 1 \\ \pm 1\end{bmatrix}
   a_{lng, max} \sqrt{1 - \left(\frac{\frac{d\theta}{ds}v_c^2}{a_{lat,max}}\right)^2} \\
\left(\frac{K_2' K_1 v_c^2}{K_2' \begin{bmatrix} 1 \\ \pm 1\end{bmatrix}a_{lng,max}}\right)^2
&= 1 - \left(\frac{\frac{d\theta}{ds}v_c^2}{a_{lat,max}}\right)^2 \\
v_c &= \min \left(
\left(\frac{K_2'K_1}{K_2' \begin{bmatrix} 1 \\ \pm 1\end{bmatrix}a_{lng, max}}\right)^2 +
   \left(\frac{\frac{d\theta}{ds}}{a_{lat,max}}\right)^2
  \right)^{-\frac14}
\end{align*}

Where the $\pm 1$ term will have opposite signs of the two coefficients of
$K_2$ have the same sign, and will have the same sign of the coefficients of
$K_2$ have opposite signs.

$K_2'$ refers to a transformed version of $K_2$ such that $K_2'K_2 = 0$ (see
below for exact definition).

\section{Calculating Feasible Voltage Velocities}

Given:

\begin{align*}
\begin{bmatrix} 1 - r_l \frac{d\theta}{ds} \\ 1 + r_r \frac{d\theta}{ds}
  \end{bmatrix} \dot{v}_c +   \begin{bmatrix} -r_l \\ r_r \end{bmatrix} \frac{d^2\theta}{ds^2} v_c^2
  &= A \begin{bmatrix} 1 - r_l \frac{d\theta}{ds} \\ 1 + r_r \frac{d\theta}{ds}
     \end{bmatrix} v_c + B \begin{bmatrix} u_l \\ u_r \end{bmatrix} \\
\abs{u} &\le V_{max} \\
\norm{B^{-1} \left(\begin{bmatrix} 1 - r_l \frac{d\theta}{ds} \\ 1 + r_r \frac{d\theta}{ds}
  \end{bmatrix} \dot{v}_c +   \begin{bmatrix} -r_l \\ r_r \end{bmatrix} \frac{d^2\theta}{ds^2} v_c^2
  - A \begin{bmatrix} 1 - r_l \frac{d\theta}{ds} \\ 1 + r_r \frac{d\theta}{ds}
     \end{bmatrix} v_c\right) }_\infty \le V_{max} \\
\end{align}

This is a linear program for fixed $v_c$. We wish to determine the limits of
$v_c$ for which the linear program will have a solution.

In order to analyze this linear program in $\dot{v}_c$, it is useful to know the
signs of the coefficients of $\dot{v}_c$, which in this case are
$B^{-1} \begin{bmatrix}1 - r_l\frac{d\theta}{ds} \\ 1 + r_r \frac{d\theta}{ds}\end{bmatrix}$

If either coefficient is zero, then for any given
$v_c$, then the linear program
becomes trivial, as the term with $\dot{v}_c$ will drop out of one of the rows
and so the other row will be guaranteed to be valid for some $\dot{v}_c$ while
the row with the dropped $\dot{v}_c$ will either be valid or not. If $B$ were
the identity matrix, then this would
correspond to situations where one wheel is stationary and the
acceleration of that wheel is not affected by the overall acceleration along the
path. Determining the limits of allowable velocities is then an issue of
calculating the point at which the stationary wheel goes from being controllable
to not, for a given overall velocity.

Otherwise, the edge of feasibility will occur
when the pair of constraints (where one pair is for the left voltage, one for
the right) cross and so we will have each voltage as $\pm V_{max}$; if
the signs of the coefficients differ, then we will have
$u_l = u_r$. Otherwise, $u_l = -u_r$.

For $u_l = u_r$, attempting to maximize $v_c$:
\begin{align*}
B^{-1} \left(\begin{bmatrix} 1 - r_l \frac{d\theta}{ds} \\ 1 + r_r \frac{d\theta}{ds}
  \end{bmatrix} \dot{v}_c +   \begin{bmatrix} -r_l \\ r_r \end{bmatrix} \frac{d^2\theta}{ds^2} v_c^2
  - A \begin{bmatrix} 1 - r_l \frac{d\theta}{ds} \\ 1 + r_r \frac{d\theta}{ds}
     \end{bmatrix} v_c\right) &= \pm \begin{bmatrix} V_{max} \\ V_{max} \end{bmatrix} \\
\begin{pmatrix} 1 + r_r\frac{d\theta}{ds} \\ r_l\frac{d\theta}{ds} - 1
  \end{pmatrix}\left(
\begin{bmatrix} 1 - r_l \frac{d\theta}{ds} \\ 1 + r_r \frac{d\theta}{ds}
  \end{bmatrix} \dot{v}_c +   \begin{bmatrix} -r_l \\ r_r \end{bmatrix} \frac{d^2\theta}{ds^2} v_c^2
  - A \begin{bmatrix} 1 - r_l \frac{d\theta}{ds} \\ 1 + r_r \frac{d\theta}{ds}
     \end{bmatrix} v_c &= \pm B \begin{bmatrix} V_{max} \\ V_{max} \end{bmatrix}
     \right) \\
(r_l \frac{d\theta}{ds} - 1) (1 + r_r \frac{d\theta}{ds}) \dot{v}_c +
  \begin{bmatrix} -r_l - r_lr_r\frac{d\theta}{ds} \\ -r_r +
  r_lr_r\frac{d\theta}{ds} \end{bmatrix} \frac{d^2\theta}{ds^2} v_c^2
  - \begin{bmatrix} 1 + r_r \frac{d\theta}{ds} & 0 \\ 0 & r_l \frac{d\theta}{ds} - 1
  \end{bmatrix} A \begin{bmatrix} 1 - r_l \frac{d\theta}{ds} \\ 1 + r_r \frac{d\theta}{ds}
     \end{bmatrix} v_c &=
   \begin{bmatrix} 1 + r_r \frac{d\theta}{ds} & 0 \\ 0 & r_l \frac{d\theta}{ds} - 1
   \end{bmatrix} B
     \begin{bmatrix} \pm V_{max} \\ \pm V_{max} \end{bmatrix}
\end{align*}

\begin{align*}
 -(r_l + r_r) \frac{d^2\theta}{ds^2} v_c^2
 - ((1 + r_r\frac{d\theta}{ds})(A_{11}(1 - r_l\frac{d\theta}{ds}) + A_{12}(1 + r_r\frac{d\theta}{ds}))
  + (r_l\frac{d\theta}{ds} - 1)(A_{12}(1 - r_l\frac{d\theta}{ds}) + A_{11}(1 + r_r\frac{d\theta}{ds}))
     ) v_c &= \\
   \pm V_{max}\frac{d\theta}{ds}(B_{12}r_l + B_{11}r_l + B_{11}r_r + B_{12}r_r) \\
 (r_l + r_r) \frac{d^2\theta}{ds^2} v_c^2 +
 \frac{d\theta}{ds}( A_{11} (-r_l - r_r) + A_{12}(r_r + r_l) +
  r_r(A_{11}(1 - r_l\frac{d\theta}{ds}) + A_{12}(1 + r_r\frac{d\theta}{ds}))  +
  r_l(A_{12}(1 - r_l\frac{d\theta}{ds}) + A_{11}(1 + r_r\frac{d\theta}{ds}))
     ) v_c &= \\
 (r_l + r_r) \frac{d^2\theta}{ds^2} v_c^2 +
 \frac{d\theta}{ds}( (r_l + r_r) (A_{12} - A_{11}) +
 A_{11}(r_l + r_r) + A_{12}(r_l + r_r) +
 A_{12}\frac{d\theta}{ds}(r_r^2 - r_l^2)) v_c &= \\
 (r_l + r_r) \left( \frac{d^2\theta}{ds^2} v_c^2 +
 \frac{d\theta}{ds}A_{12}\left(2 +
 \frac{d\theta}{ds}(r_r - r_l)\right) v_c\right) &= \\
     \pm V_{max}(r_l + r_r)\frac{d\theta}{ds}(B_{11} + B_{12}) \\
 \frac{d^2\theta}{ds^2} v_c^2 +
 \frac{d\theta}{ds}A_{12}\left(2 + \frac{d\theta}{ds}(r_r - r_l)\right) v_c =
     \pm V_{max}\frac{d\theta}{ds}(B_{11} + B_{12}) \\
\end{align*}

Note that when I eliminate parts of $A$ and $B$ I am taking advantage of the
fact that different rows of $A$ and $B$ are just reordered version of one
another (i.e., each is of the format $\begin{bmatrix} a & b \\ b & a
\end{bmatrix}$).

In the final equation, we then execute the quadratic formula for $v_c$.
Note that if you write
the quadratic formula out there will be up to 4 real solutions (the two
solutions to the quadratic formula by the two $\pm V_{max}$ results); because
of the placement of the $\pm V_{max}$, then
0, 1, or 3 of the solutions will be greater than zero \footnote{The basic format of
the quadratic formula
in this case will be $b \pm \sqrt{b^2 \pm k}$ or some equivalent, and
$0 < \abs{k} \le b^2$ ($\abs{k} \le b^$ is necessary to have 4 real solutions;
and $k = 0$ only if
$b$ also is zero, or if $V_{max} = 0$, which are not particularly
relevant corner cases). If $b$ is positive,
then 3 will be positive ($b +$ anything is positive, and since
$\sqrt{b^2 - k} \le b$, $b - \sqrt{b^2 - k} \ge 0$, while $b - \sqrt{b^2 + k}
\le 0$). A similar argument holds for negative $b$. If $b = 0$, then if there
are four real solutions, all four are zero. If there are not four real
solutions, then $\abs{k} > b^2$ and we have two real solutions. In this case,
because $\sqrt{b^2 + k} > b$, we will have one positive and one negative
solution.}. The zero positive solution case is not particularly interesting, as
it only applies at zero curvature (which cannot apply here, as we already stated
that $\abs{r\frac{d\theta}{ds}} > 1$ (or, if we have zero $V_{max}$). The one
or three provide us our cutoffs for where there may be a gap in allowable
accelerations.

For situations where the coefficients have the same sign, attempting to maximize $v_c$:

\begin{align*}
(r_l \frac{d\theta}{ds} - 1) (1 + r_r \frac{d\theta}{ds}) \dot{v}_c +
  \begin{bmatrix} -r_l - r_lr_r\frac{d\theta}{ds} \\ -r_r +
  r_lr_r\frac{d\theta}{ds} \end{bmatrix} \frac{d^2\theta}{ds^2} v_c^2
  - \begin{bmatrix} 1 + r_r \frac{d\theta}{ds} & 0 \\ 0 & r_l \frac{d\theta}{ds} - 1
  \end{bmatrix} A \begin{bmatrix} 1 - r_l \frac{d\theta}{ds} \\ 1 + r_r \frac{d\theta}{ds}
     \end{bmatrix} v_c &=
   \begin{bmatrix} 1 + r_r \frac{d\theta}{ds} & 0 \\ 0 & r_l \frac{d\theta}{ds} - 1
   \end{bmatrix} B
     \begin{bmatrix} \pm V_{max} \\ \mp V_{max} \end{bmatrix}
\end{align*}

\begin{align*}
 (r_l + r_r) \left( \frac{d^2\theta}{ds^2} v_c^2 +
 \frac{d\theta}{ds}A_{12}\left(2 +
 \frac{d\theta}{ds}(r_r - r_l)\right) v_c\right) &=
   \pm V_{max}(B_{11} - B_{12})(2 + \frac{d\theta}{ds}(r_r - r_l)) \\
 \frac{d^2\theta}{ds^2} v_c^2 +
 \frac{d\theta}{ds}A_{12}\left(2 +
 \frac{d\theta}{ds}(r_r - r_l)\right) v_c &=
   \pm V_{max}\frac{(B_{11} - B_{12})(2 + \frac{d\theta}{ds}(r_r - r_l))}{r_r + r_l}\\
\end{align*}

Where the same basic rules apply to the quadratic formula solutions.

Considering situations where one of the rows in the equations disappear,
we end up with the $\dot{v}_c$ term disappearing and
we must solve the following equation for whichever row of $B^{-1}$ $B^{-1}_N$
causes the $\dot{v}_c$ term to disappear:
\begin{align*}
B^{-1} &= \frac{1}{B_{11}^2 - B_{12}^2}\begin{bmatrix} B_{11} & -B_{12} \\
                                   -B_{12} & B_{11} \end{bmatrix} \\
B^{-1}_N\left(
  \begin{bmatrix} -r_l \\ r_r \end{bmatrix} \frac{d^2\theta}{ds^2} v_c^2
  - A \begin{bmatrix} 1 - r_l \frac{d\theta}{ds} \\
                      1 + r_r \frac{d\theta}{ds} \end{bmatrix} v_c \right)
                      &= \pm V_{max} \\
\end{align*}


Trying to be more general:
\begin{align*}
B^{-1} \left(\begin{bmatrix} 1 - r_l \frac{d\theta}{ds} \\ 1 + r_r \frac{d\theta}{ds}
  \end{bmatrix} \dot{v}_c +   \begin{bmatrix} -r_l \\ r_r \end{bmatrix} \frac{d^2\theta}{ds^2} v_c^2
  - A \begin{bmatrix} 1 - r_l \frac{d\theta}{ds} \\ 1 + r_r \frac{d\theta}{ds}
     \end{bmatrix} v_c\right) &= \pm \begin{bmatrix} V_{max} \\ V_{max} \end{bmatrix} \\
B^{-1} \left(K_2 \dot{v}_c +   K_1 v_c^2 - A K_2 v_c\right) &=
   \pm \begin{bmatrix} V_{max} \\ V_{max} \end{bmatrix} \\
K_2' &= K_2^T  \begin{bmatrix} 0 & 1 \\ -1 & 0 \end{bmatrix} \\
K_2' \left(K_2 \dot{v}_c +   K_1 v_c^2 - A K_2 v_c\right) &=
   K_2' B \begin{bmatrix} \pm 1 \\ \pm 1 \end{bmatrix} V_{max} \\
K_2' \left(K_2 \dot{v}_c +   K_1 v_c^2 - A K_2 v_c\right) &=
   K_2' B \begin{bmatrix} \pm 1 \\ \pm 1 \end{bmatrix} V_{max} \\
K_2' \left(K_1 v_c^2 - A K_2 v_c\right) &=
   K_2' B \begin{bmatrix} \pm 1 \\ \pm 1 \end{bmatrix} V_{max} \\
 \end{align*}

The quadratic formula coefficients:
\begin{align*}
a v_c^2 + b v_c + c &= 0 \\
a &= K_2'K_1 \\
b &= -K_2'AK_2 \\
c &= -K_2' B \begin{bmatrix} \pm 1 \\ \pm 1 \end{bmatrix} V_{max} \\
b^2 - 4 ac &= (K_2'AK_2)^2 + 4K_2'K_1 K_2'B \begin{bmatrix} \pm 1 \\ \pm 1 \end{bmatrix} V_{max} \\
\end{align*}

%Interestingly, the solutions for both basically devolve to:
%\begin{align*}
% \frac{d^2\theta}{ds^2} v_c^2 + 2 \frac{d\theta}{ds} A_{12} v_c
%  = \pm V_{max}\max(\abs{\frac{d\theta}{ds}}, 1)(B_{11} + B_{12}) \\
%\end{align*}
%
%While not overly interesting, the moments when we will have a split in
%feasibility will be at:
%
%\begin{align*}
%0 < \abs{4V_{max}\max(\abs{\frac{d\theta}{ds}}, 1)(B_{11} +
%B_{12})\frac{d^2\theta}{ds^2}} &\le \abs{2\frac{d\theta}{ds} A_{12}} \\
%\abs{\frac{d\theta}{ds}} &\le 1 : \\
%0 < \abs{2V_{max}(B_{11} + B_{12})\frac{d^2\theta}{ds^2}} &\le
%\abs{\frac{d\theta}{ds} A_{12}} \\
%\abs{\frac{d\theta}{ds}} &> 1 : \\
%0 < \abs{2V_{max}(B_{11} + B_{12})\frac{d^2\theta}{ds^2}} &\le \abs{A_{12}} \\
%\end{align*}
%
%Note that, in general, $A_{12}$ will be relatively small when compared to
%$2V_{max}(B_{11} + B_{12})$.

%\begin{align*}
%B_{12} = B_{21} &, B_{11} = B_{22} \\
%2 r \frac{d^2\theta}{ds^2} v_c^2 &= \pm 2V_{max}r\frac{d\theta}{ds} (B_{12} - B_{11}) \\
%v_c &= \sqrt{\pm\frac{V_{max}\frac{d\theta}{ds}}{\frac{d^2\theta}{ds^2}} (B_{12} - B_{11})} \\
%\end{align*}
%
%\begin{align*}
%\begin{bmatrix} 1 - r^2 \frac{d\theta}{ds}^2 \\ 1 - r^2 \frac{d\theta}{ds}^2
%  \end{bmatrix} \dot{v}_c +   \begin{bmatrix} -1 - r\frac{d\theta}{ds} \\ 1 -
%  r\frac{d\theta}{ds} \end{bmatrix} r \frac{d^2\theta}{ds^2} v_c^2
%  - A \begin{bmatrix} 1 - r^2 \frac{d\theta}{ds}^2 \\ 1 - r^2 \frac{d\theta}{ds}^2
%     \end{bmatrix} v_c &=  B
%     \begin{bmatrix} \mp V_{max} (1 + r\frac{d\theta}{ds}) \\ \pm V_{max} (1 -
%     r\frac{d\theta}{ds}) \end{bmatrix} \\
% 2 r \frac{d^2\theta}{ds^2} v_c^2
%      &= \pm V_{max}(B_{21} - B_{22} - B_{11} + B_{12}) \\
% r \frac{d^2\theta}{ds^2} v_c^2
%      &= \pm V_{max}(B_{12} - B_{11}) \\
%v_c &= \sqrt{\pm \frac{V_{max}}{r\frac{d^2\theta}{ds^2}} (B_{12} - B_{11})} \\
%\end{align*}

\end{document}
